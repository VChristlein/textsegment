\documentclass[a4paper, 9pt, twocolumn]{extarticle}

\usepackage{amsmath,graphicx,amssymb,cite}
\usepackage[utf8]{inputenc}
\usepackage[english]{babel}
\usepackage{url}


\addtolength{\textwidth}{2.1cm}
\addtolength{\topmargin}{-2.4cm}
\addtolength{\oddsidemargin}{-1.1 cm}
\addtolength{\textheight}{4.5cm}
\setlength{\columnsep}{0.7cm}

\pagestyle{empty}

\begin{document}

\date{\normalsize 11.04.2018}

\title{\vspace{-8mm}\textbf{\Large
Text binarization for historical documents with Tensorflow\footnote{This is the
  summary of my computer science master project \emph{Projekt Mustererkennung
  (ProjME)}, Winter Term 2017/18, Friedrich-Alexander Universit\"at
  Erlangen-N\"urnberg.
  Supervisor: Vincent Christ\-lein.
}}}

\author{
{
\begin{minipage}{\textwidth}
\center
Hendrik Schröter\\
\small
ko01jaxu\\
Friedrich-Alexander Universit\"at Erlangen-N\"urnberg
\protect\\{} %
\url{hendrik.m.schroeter@fau.de}
\end{minipage}
}
}

\maketitle
\thispagestyle{empty}

%%%%%%%%%%%%%%%%%%%%%%%%%%%%%%%%%%%%%%%%%%%%%%%%%%%%%%%%%%%%%%%%%%%%%%%%%%%%%%
\section{Introduction}
\label{section:introduction}

The task of this project was to implement and combine state of the art methods
for text binarization on the DIBCO dataset \cite{pratikakis2017icdar2017}. The
Tensorflow \cite{abadi2016tensorflow} based implementation uses convolutional
neural networks to generate ground truth text images for a given document
image. \\
The underlying model architecture is an U-Net \cite{ronneberger2015u}
followed by a conditional random field as recurrent neural network (CRF as RNN)
\cite{crfasrnn_ICCV2015} included in the end-to-end deep learning pipeline.

%%%%%%%%%%%%%%%%%%%%%%%%%%%%%%%%%%%%%%%%%%%%%%%%%%%%%%%%%%%%%%%%%%%%%%%%%%%%%%

\section{Approach}
\label{section:approach}

The U-Net implementation was first tested on the popular CIFAR10 image
segmentation challenge to check if the implementation works correctly.
Furthermore an implementation of the CRF as RNN was added from
\cite{Jayasumana2017}.\\
In the next step I trained a model on the DIBCO dataset. Since the DIBCO has
only a few images (86 images for train) data augmentation was implemented. The
model architecture was also evaluated on the HISDB dataset which does not model
the text itself but a polygon bounding area around each line. Later I tried to
use transfer learning from the HISDB to DIBCO.

%%%%%%%%%%%%%%%%%%%%%%%%%%%%%%%%%%%%%%%%%%%%%%%%%%%%%%%%%%%%%%%%%%%%%%%%%%%%%%

\section{Implementation}
\label{section:implementation}

\subsection*{Data loading}

I prepared the HISDB and DIBCO datasets by combining all images of all
challenges into one package and provided two csv files which list the file paths
of the training respectively test images. When starting training, the dataset
will get downloaded automatically and packed into a tfrecord for efficient data
loading. The images will then be unpacked, preprocessed, shuffled and packed
into batches tf.data.Dataset api. The preprocessing includes normalization using
a precomputed mean of the train dataset, per image standard deviation, random
crop of a patch for training (e.g. 250x250 pixel), random rotation, random
contrast and brightness changes and rgb to bgr conversion.

\subsection*{U-Net}

The U-Net architecture was chosen because it because it provides on the one hand
short cut connections similar to a Resnet architecture. This provides a better
gradient flow and faster and more stable training. Ont the other hand the U-Net
provides a relatively scale invariant processing due to the down and upscaling
layers. Therefor the U-Net can handle different sizes of text automatically. I
tested U-Nets of depth 3 to 5 with respectively 3 or 5 down and upscale blocks.
One block consists out of 2 x [convolution, dropout, batchnorm and relu
activation] and a max pool layer for a downscale block and a transposed
convolution for a upscale block. The Output from each downscale before the max
pooling is stacked with the input of the corresponding upscale layer. The
network was trained using a momentum optimizer with an initial learning rate of
$0.1\cdot \text{batch\_size} / 64$ and a momentum of 0.9.

\subsection*{CRF}

The CRF is implemented as RNN formulation of the mean field algorithm. However I
found that, since only are learned $3\cdot N^2$ (where $N$: number of classes)
parameters the CRF does bring a significant improvement of the accuracy.

\subsection*{Transfer learning}

Since the DIBCO dataset is very small, I tried to improve the accuracy using
pretraining on the HISDB dataset. Therefor I trained the network on the HISDB
dataset and used the pretrained weights as initialization for the DIBCO
training. However I found that the network already overfitted on the HISDB
ground truth images (polygon around the line) and was not able to generalize
using the pretrained features. The result was still a predicted polygon for the
DIBCO dataset but much smaller around each word or letter. Also the next attempt
using only the weights of the downscale layers as initialization didn't bring a
accuracy boost over only training with the DIBCO data. I think this was the case
because of the very differing ground truth images of both datasets.

%%%%%%%%%%%%%%%%%%%%%%%%%%%%%%%%%%%%%%%%%%%%%%%%%%%%%%%%%%%%%%%%%%%%%%%%%%%%%%
\bibliographystyle{abbrv}
\small
\bibliography{references}
%%%%%%%%%%%%%%%%%%%%%%%%%%%%%%%%%%%%%%%%%%%%%%%%%%%%%%%%%%%%%%%%%%%%%%%%%%%%%%

\end{document}
